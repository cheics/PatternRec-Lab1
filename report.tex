\documentclass[article, 1.5space, letterpaper, 12pt, oneside, header, footer]{SydeClass}
\graphicspath{{images/}}
\usepackage{subfigure}
\usepackage{amsmath}
\usepackage{eqnarray}
\usepackage{rotating}

\usepackage{array}
\usepackage{multirow}




% --------- Title Info -----------
\titlestyle{design} % used in SydeTitle.tex. Can equal one of the following values: design, work

\title{Lab 1}
\subtitle{Clusters and Classification Boundaries}

\coursecode{SYDE 472}
\department{Systems Design Engineering}

\author{Colin Heics, 20240543}
\authorheader{C. Heics}
\authortwo{Rob Sparrow, 20275155}
\authorheadertwo{R. Sparrow}
\authorthree{Philip Wang, STUNUM}
\authorheaderthree{P. Wang}

\date{\today}
\instructor{Alex Wong}

\subsectionfont{\normalsize}
\setcounter{secnumdepth}{2}
\setcounter{tocdepth}{1}

\input{matlabFormating}

% ############  ############
\begin{document}

% ---------- Title ------------
\input{SydeTitle}

% ############ Chapters ############
\pagenumbering{arabic}

\section{Generating Clusters}

\section{Introduction}
This lab investigates three areas: calculating orthonormal transformations, creating decision boundaries using different classification methods, and assessing classification error associated with different methods. All calculations for the purpose of this lab are carried out using the MatLab program.

First, data for five separate classes is generated using bivariate Gaussian distribution parameters provided in the lab description. This results in five clusters of two-dimensional data being generated, based on the underlying statistics specified, which is analyzed with respect to the unit standard deviation contour in each class. Analysis of the data is separated into two cases: Case 1 corresponding to the first two classes, and Case 2 corresponding to the remaining three.

Next, the cluster data generated is used in development of five separate classification methods to allow for the plotting of decision boundaries in each case. First, the Mean Euclidean Distance (MED), Generalized Euclidean Distance (GED), and Maximum A Posteriori (MAP) methods are implemented and applied to all classes of data. Decision boundaries are plotted for all three methods using data from each of the Cases developed previously, to allow for comparison of the boundaries. Next, the Nearest Neighbor (NN) and Five Nearest Neighbor (5NN) methods are applied to all classes of data. The decision boundaries for both methods are plotted again for the Case 1 and Case 2 sets of classes to allow for comparison.

Lastly, the experimental error rate and confusion matrices are developed for each method applied to both Case 1 and Case 2. The experimental error and confusion matrices are analyzed to allow for comparison between the experimental results of each method.

\section{Generating Clusters}


\section{SAMPLE IMAGE CODEZ}
\begin{figure}[ht]
\centering
	\subfigure[Base Image]{
	\includegraphics[width=0.45\linewidth]{question2/original_image}
	}
	\subfigure[Luma Component]{
	\includegraphics[width=0.45\linewidth]{question2/luma}
	}
	\caption{YCrCb channels of pepper image}
	\label{fig:noiseGeneration.toy}
\end{figure}
 	
\clearpage

\begin{figure}[ht]
\centering	
	\includegraphics[width=0.65\linewidth]{question2/luma_subsampled}
	\caption{Luma subsampled image}
\end{figure}



\section{Classifiers}

\subsection{Implementation}

\subsubsection{Mean Euclidean Distance}

\subsubsection{General Euclidean Distance}
The GED classifier is implemented in a similar manner as the MED classifier. However, in the case of GED a whitening transform is applied to the samples to transform them onto a space where features are both uncorrelated, and have unit-variances. This is accomplished using the weighting matrix W. The distance between two points in the transformed space is calculated as,


\begin{eqnarray}
\label{eqn:GED-whitening}
\left [ W=\Lambda^{-1/2}\Phi^{T}  \right ]
\end{eqnarray}



In the above equation, $\Lambda$ contains the eigen values of the covariance matrix $\Sigma$ as elements, and $\Phi$ contains the eigenvectors of $\Sigma$. The simplified distance function is included in \ref{eqn:GED}.

\begin{eqnarray}
\label{eqn:GED}
{d}_{G}(x,z) = {\left [ (x-z)^{T}\Phi\Lambda^{-1/2}\Phi^{T}(x-z) \right ]}^{1/2}
\end{eqnarray}


The decision boundary is therefore calculated in the ttwo- and three- class cases as,

\begin{eqnarray}
\label{eqn:boundary-GED}
& d_{E} (x,z_{1}) = d_{E} (x,z_{2}) \\
& \left [ (x-{z}_{1})^{T}\Phi\Lambda^{-1/2}\Phi^{T}(x-z_{1}) \right ]^{1/2} \\
= & \left [ (x-z_{2})^{T}\Phi\Lambda^{-1/2}\Phi^{T}(x-z_{2}) \right ]^{1/2}  \nonumber \\
&\left [ (x-{z}_{1})^{T}\Phi\Lambda^{-1/2}\Phi^{T}(x-z_{1}) \right ]^{1/2} \\
= &\left [ (x-z_{2})^{T}\Phi\Lambda^{-1/2}\Phi^{T}(x-z_{2}) \right ]^{1/2}  \nonumber \\
= &\left [ (x-z_{3})^{T}\Phi\Lambda^{-1/2}\Phi^{T}(x-z_{3}) \right ]^{1/2}  \nonumber
\end{eqnarray}



In the case of the MatLab implementation, all points on the grid are classified based on identifying the minimum distance between the point and the mean of each class in the transformed space. This allows for a simple contour to be plotted showing the decision boundary between each class. This is shown below for both the two- and three- class case,

\begin{eqnarray}
\label{eqn:pointClass-GED}
min(d_{E} (x,z_{1}), d_{E} (x,z_{2})) \\
min(d_{E} (x,z_{1}), d_{E} (x,z_{2}), d_{E} (x,z_{3}))
\end{eqnarray}


This implementation of the GED classifier and method for creating the decision boundary is shown in Appendix A.

\subsubsection{Maximum A Posteriori}

\subsubsection{Nearest Neighbor}

\subsubsection{Five Nearest Neighbor}

\subsection{Results}
All classification methods were applied in each of Case 1 and Case 2. To aid in analysis, MED, GED, and MAP classifier boundaries are all plotted together in one figure, along with data clusters and unit standard deviation contours. In the Case 1 scenario (Figure x), the GED and MAP classification boundaries, shown as magenta and blue lines respectively, lie on top of each other with the magenta line being obscured. This is because the MAP classifier is reduced to GED in this case, as the a priori probabilities for each class are equal. The classification boundaries are essentially two very slightly curved, but relatively straight, lines separating the two data clusters. The MED classification boundary is represented by the steeper straight cyan line. This makes sense intuitively, as the classification boundary represents the perpendicular bisector of the mean in each cluster. In the MED case only the mean, and not the covariance information for the two classes, is considered.

In the Case 2 scenario (Figure x), the MED classification boundary is shown as the light blue lines. The MED classification boundary is shown to be three straight lines, interescting near the mid-point between the three data clusters. The MED classifier does not take into account the covariance matrices of the three data clusters, so the performance described is as expected. The GED classification boundary is shown as the contoured magenta lines in the figure, intersecting near the midpoint between the three classes of data. This is an intuitive result, as the classification boundary better wraps around the unit standard deviation contours for three data classes. The MAP decision boundary is shown as the blue lines, slightly offset to the left of the GED decision boundary. In this case the decision boundaries do not overlay each other as in the Case 1 scenario. This is a result of the a priori probabilities for each of the three classes being different in this case, with the probability information altering the performance of the classification method.

Next, the NN and 5NN classification boundaries were plotted together for both the Case 1 and Case 2 data sets, along with unit variance contours for each class, to allow for comparison. The decision boundaries for Case 1 (Figure x) are fairly similar for both methods, with the decision boundaries being shown as jagged lines separating the data. The key difference in the two methods is that the 5NN classification method does not result in classification boundaries around outliers of the two data sets.

For the Case 2 scenario (Figure x), performance of the two methods was similar. Again, the sensitivity of the NN method to outliers is seen to result to decision boundaries encircling outliers. The 5NN classifier is not as prone to these outliers, resulting in a more intuitive decision boundary between the three classes of data.

\section{Error Analysis}
\subsection{Experimental error rate}
To find the experimental error rate $P(\epsilon)$ for Case 1, a discretized form of the following equation was used.
\begin{eqnarray}
\label{eqn:2-class_cont_PofE}
P(\epsilon)=\int_{R_A}P(\underbar {x}|B)P(B)d\underbar {x} +
            \int_{R_B}P(\underbar {x}|A)P(A)d\underbar {x}
\end{eqnarray}

The discretized form becomes
\begin{eqnarray}
\label{eqn:2-class_disc_PofE}
P(\epsilon)\approx\sum_{i,j\ in\ R_A}P(\underbar {x}|B)_{i,j}P(B)wh +
                  \sum_{i.j\ in\ R_B}P(\underbar {x}|A)_{i,j}P(A)wh
\end{eqnarray}

where $w$ and $l$ are the width and height of the discretized points. The MAP classification boundary that were found previously were used for determining $R_A$ and $R_B$. Although the equation describes summing values over an infinite number of points, a finite set of points had to be selected to feasibly perform the calculation. This introduces some negligible error. The same number of points used to calculate the classification data was used to calculate $P(\epsilon)$. The result was $P(\epsilon)\approx0.06281$.

To find $P(\epsilon)$ for Case 2, the same method was used as in Case 1 except that the equation for $P(\epsilon)$ had to be modified for three classes.
\begin{eqnarray}
\label{eqn:3-class_cont_PofE}
P(\epsilon)& = \int_{R_C}[P(\underbar{x}|D)P(D)+P(\underbar{x}|E)P(E)]d\underbar{x} \\
           & + \int_{R_D}[P(\underbar{x}|C)P(C)+P(\underbar{x}|E)P(E)]d\underbar{x} \nonumber\\
           & + \int_{R_E}[P(\underbar{x}|C)P(C)+P(\underbar{x}|D)P(D)]d\underbar{x}\nonumber
\end{eqnarray}

The discretized form of this equation becomes
\begin{eqnarray}
\label{eqn:3-class_disc_PofE}
P(\epsilon)&\approx \sum_{i,j\ in\ R_C}[P(\underbar{x}|D)_{i,j}P(D)+P(\underbar{x}|E)_{i,j}P(E)]wh \\
           & +      \sum_{i,j\ in\ R_D}[P(\underbar{x}|C)_{i,j}P(C)+P(\underbar{x}|E)_{i,j}P(E)]wh \nonumber\\
           & +      \sum_{i,j\ in\ R_E}[P(\underbar{x}|C)_{i,j}P(C)+P(\underbar{x}|D)_{i,j}P(D)]wh \nonumber
\end{eqnarray}

The result for Case 2 was found to be $P(\epsilon)\approx0.1686$.

Case 1 was found to have a much lower experimental error rate than Case 2. This was expected because Case 2 had one more class than Case 1 and because the unit standard deviation contours of the clusters of Case 2 were much closer together than the contours of the clusers of Case 1.
 
\subsection{Confusion matrices for GED, MED and MAP}

A confusion matrix shows the performance of a classifier on a set of data. It compares the true values of the data against the predicted classifications by the classifier, and attempts to quantify the types of errors being made.

For the GED, MED and MAP classifiers, the same data used to generate the classifier was used to test the classifier.

\subsubsection{GED}
\begin{figure}[!ht]
\begin{minipage}[b]{0.5\linewidth}
\centering
	\begin{tabular}{ccc|c|c}
	 & &\multicolumn{2}{c}{Predicted} &\\
	  & & \bf{A} &  \bf{B} & total \\
	 \cline{3-5}
	 \multirow{2}{*}{\begin{sideways}Actual\end{sideways}} & \bf{A'}& 186 & 14 & 200 \\
	 \cline{3-5}
	 & \bf{B'}& 10 & 190 & 200 \\
	  \cline{3-5}
	 &total&196&204&\\
	\end{tabular}
\end{minipage}
\hspace{0.5cm}
\begin{minipage}[b]{0.5\linewidth}
	\begin{tabular}{r|c}
	\hline
	Correct& 376\\
	Total& 400\\
	\hline
	\% Correct& 94\%\\
	\hline
	\end{tabular}
\end{minipage}
\vspace{1mm}
\caption{GED performance on A and B}
\end{figure}

\begin{figure}[!ht]
\begin{minipage}[b]{0.5\linewidth}
\centering
	\begin{tabular}{ccc|c|c|c}
	 & &\multicolumn{3}{c}{Predicted} &\\
	  & & \bf{C} &  \bf{D} & \bf{E} & total \\
	 \cline{3-6}
	 \multirow{3}{*}{\begin{sideways}Actual\end{sideways}} & \bf{C'}& 91 & 0 & 9 & 100\\
	 \cline{3-6}
	 & \bf{D'}& 6 & 164 & 30 & 200\\
	  \cline{3-6}
	 & \bf{E'}& 29 & 11 & 110 &  150\\
	  \cline{3-6}
	 &total&126&175&149\\
	\end{tabular}
\end{minipage}
\hspace{0.5cm}
\begin{minipage}[b]{0.5\linewidth}
	\begin{tabular}{r|c}
	\hline
	Correct& 365\\
	Total& 450\\
	\hline
	\% Correct& 81.1\%\\
	\hline
	\end{tabular}
\end{minipage}
\vspace{1mm}
\caption{GED performance on C, D and E}
\end{figure}

\clearpage

\subsubsection{MED}
\begin{figure}[!ht]
\begin{minipage}[b]{0.5\linewidth}
\centering
	\begin{tabular}{ccc|c|c}
	 & &\multicolumn{2}{c}{Predicted} &\\
	  & & \bf{A} &  \bf{B} & total \\
	 \cline{3-5}
	 \multirow{2}{*}{\begin{sideways}Actual\end{sideways}} & \bf{A'}& 184 & 16 & 200 \\
	 \cline{3-5}
	 & \bf{B'}& 10 & 190 & 200 \\
	  \cline{3-5}
	 &total&196&204&\\
	\end{tabular}
\end{minipage}
\hspace{0.5cm}
\begin{minipage}[b]{0.5\linewidth}
	\begin{tabular}{r|c}
	\hline
	Correct& 374\\
	Total& 400\\
	\hline
	\% Correct& 93.5\%\\
	\hline
	\end{tabular}
\end{minipage}
\vspace{1mm}
\caption{MED performance on A and B}
\end{figure}


\begin{figure}[!ht]
\begin{minipage}[b]{0.5\linewidth}
\centering
	\begin{tabular}{ccc|c|c|c}
	 & &\multicolumn{3}{c}{Predicted} &\\
	  & & \bf{C} &  \bf{D} & \bf{E} & total \\
	 \cline{3-6}
	 \multirow{3}{*}{\begin{sideways}Actual\end{sideways}} & \bf{C'}& 70 & 1 & 29 & 100\\
	 \cline{3-6}
	 & \bf{D'}& 9 & 167 & 24 & 200\\
	  \cline{3-6}
	 & \bf{E'}& 27 & 13 & 110 &  150\\
	  \cline{3-6}
	 &total&106&181&163\\
	\end{tabular}
\end{minipage}
\hspace{0.5cm}
\begin{minipage}[b]{0.5\linewidth}
	\begin{tabular}{r|c}
	\hline
	Correct& 347\\
	Total& 450\\
	\hline
	\% Correct& 77.1\%\\
	\hline
	\end{tabular}
\end{minipage}
\vspace{1mm}
\caption{MED performance on C, D and E}
\end{figure}

\subsubsection{MAP}
\begin{figure}[!ht]
\begin{minipage}[b]{0.5\linewidth}
\centering
	\begin{tabular}{ccc|c|c}
	 & &\multicolumn{2}{c}{Predicted} &\\
	  & & \bf{A} &  \bf{B} & total \\
	 \cline{3-5}
	 \multirow{2}{*}{\begin{sideways}Actual\end{sideways}} & \bf{A'}& 186 & 14 & 200 \\
	 \cline{3-5}
	 & \bf{B'}& 10 & 190 & 200 \\
	  \cline{3-5}
	 &total&196&204&\\
	\end{tabular}
\end{minipage}
\hspace{0.5cm}
\begin{minipage}[b]{0.5\linewidth}
	\begin{tabular}{r|c}
	\hline
	Correct& 376\\
	Total& 400\\
	\hline
	\% Correct& 94\%\\
	\hline
	\end{tabular}
\end{minipage}
\vspace{1mm}
\caption{MAP performance on A and B}
\end{figure}


\begin{figure}[!ht]
\begin{minipage}[b]{0.5\linewidth}
\centering
	\begin{tabular}{ccc|c|c|c}
	 & &\multicolumn{3}{c}{Predicted} &\\
	  & & \bf{C} &  \bf{D} & \bf{E} & total \\
	 \cline{3-6}
	 \multirow{3}{*}{\begin{sideways}Actual\end{sideways}} & \bf{C'}& 79 & 0 & 21 & 100\\
	 \cline{3-6}
	 & \bf{D'}& 4 & 174 & 22 & 200\\
	  \cline{3-6}
	 & \bf{E'}& 23 & 20 & 107 &  150\\
	  \cline{3-6}
	 &total&106&194&150\\
	\end{tabular}
\end{minipage}
\hspace{0.5cm}
\begin{minipage}[b]{0.5\linewidth}
	\begin{tabular}{r|c}
	\hline
	Correct& 360\\
	Total& 450\\
	\hline
	\% Correct& 80\%\\
	\hline
	\end{tabular}
\end{minipage}
\vspace{1mm}
\caption{MAP performance on C, D and E}
\end{figure}

\clearpage

\subsection{NN and 5NN}

For the NN and 5NN classifiers, the classifiers were developed from a set of training data, and the classifier was tested with a new set of data, distributed with the same underlying statistics as the training data. This data set was of the same size as the training data set.

\subsubsection{NN}
\begin{figure}[!ht]
\begin{minipage}[b]{0.5\linewidth}
\centering
	\begin{tabular}{ccc|c|c}
	 & &\multicolumn{2}{c}{Predicted} &\\
	  & & \bf{A} &  \bf{B} & total \\
	 \cline{3-5}
	 \multirow{2}{*}{\begin{sideways}Actual\end{sideways}} & \bf{A'}& 83 & 17 & 100 \\
	 \cline{3-5}
	 & \bf{B'}& 21 & 179 & 200 \\
	  \cline{3-5}
	 &total&104&196&\\
	\end{tabular}
\end{minipage}
\hspace{0.5cm}
\begin{minipage}[b]{0.5\linewidth}
	\begin{tabular}{r|c}
	\hline
	Correct& 279\\
	Total& 400\\
	\hline
	\% Correct& 69.8\%\\
	\hline
	\end{tabular}
\end{minipage}
\vspace{1mm}
\caption{Nearest Neighbour performance on A and B}
\end{figure}


\begin{figure}[!ht]
\begin{minipage}[b]{0.5\linewidth}
\centering
	\begin{tabular}{ccc|c|c|c}
	 & &\multicolumn{3}{c}{Predicted} &\\
	  & & \bf{C} &  \bf{D} & \bf{E} & total \\
	 \cline{3-6}
	 \multirow{3}{*}{\begin{sideways}Actual\end{sideways}} & \bf{C'}& 63 & 2 & 35 & 100\\
	 \cline{3-6}
	 & \bf{D'}& 3 & 160 & 37 & 200\\
	  \cline{3-6}
	 & \bf{E'}& 20 & 28 & 102 &  150\\
	  \cline{3-6}
	 &total&86&190&141\\
	\end{tabular}
\end{minipage}
\hspace{0.5cm}
\begin{minipage}[b]{0.5\linewidth}
	\begin{tabular}{r|c}
	\hline
	Correct& 325\\
	Total& 450 \\
	\hline
	\% Correct& 72.2\%\\
	\hline
	\end{tabular}
\end{minipage}
\vspace{1mm}
\caption{Nearest Neighbour performance on C, D and E}
\end{figure}

\clearpage

\subsubsection{5NN}
\begin{figure}[!ht]
\begin{minipage}[b]{0.5\linewidth}
\centering
	\begin{tabular}{ccc|c|c}
	 & &\multicolumn{2}{c}{Predicted} &\\
	  & & \bf{A} &  \bf{B} & total \\
	 \cline{3-5}
	 \multirow{2}{*}{\begin{sideways}Actual\end{sideways}} & \bf{A'}& 94 & 6 & 100 \\
	 \cline{3-5}
	 & \bf{B'}& 34 & 166 & 200 \\
	  \cline{3-5}
	 &total&128&172\\
	\end{tabular}
\end{minipage}
\hspace{0.5cm}
\begin{minipage}[b]{0.5\linewidth}
	\begin{tabular}{r|c}
	\hline
	Correct& 260\\
	Total& 300\\
	\hline
	\% Correct& 86.7\%\\
	\hline
	\end{tabular}
\end{minipage}
\vspace{1mm}
\caption{5 Nearest Neighbour performance on A and B}
\end{figure}

	
\begin{figure}[!ht]
\begin{minipage}[b]{0.5\linewidth}
\centering
	\begin{tabular}{ccc|c|c|c}
	 & &\multicolumn{3}{c}{Predicted} &\\
	  & & \bf{C} &  \bf{D} & \bf{E} & total \\
	 \cline{3-6}
	 \multirow{3}{*}{\begin{sideways}Actual\end{sideways}} & \bf{C'}& 75 & 0 & 25 & 100\\
	 \cline{3-6}
	 & \bf{D'}& 1 & 181 & 18 & 200\\
	  \cline{3-6}
	 & \bf{E'}& 14 & 24 & 112 &  150\\
	  \cline{3-6}
	 &total&90&205&155\\
	\end{tabular}
\end{minipage}
\hspace{0.5cm}
\begin{minipage}[b]{0.5\linewidth}
	\begin{tabular}{r|c}
	\hline
	Correct& 368\\
	Total& 450\\
	\hline
	\% Correct& 81.8\%\\
	\hline
	\end{tabular}
\end{minipage}
\vspace{1mm}
\caption{5 Nearest Neighbour performance on C, D and E}
\end{figure}

The confusion matrices of Case 2 show that points that actually belong to cluster E are the least likely to be classified correctly. Similar to why the experimental error rate for Case 2 was lower than Case 1, the greater difficulty in correctly classifying points of cluster E is because of how close cluster E is to the other two clusters. Cluster E is partially between clusters C and D. The GED and MAP classifier boundaries show this close proximity very well.
 
\clearpage


\section{Conclusions}

\section{Generating Clusters}


\section{SAMPLE IMAGE CODEZ}
\begin{figure}[ht]
\centering
	\subfigure[Base Image]{
	\includegraphics[width=0.45\linewidth]{question2/original_image}
	}
	\subfigure[Luma Component]{
	\includegraphics[width=0.45\linewidth]{question2/luma}
	}
	\caption{YCrCb channels of pepper image}
	\label{fig:noiseGeneration.toy}
\end{figure}
 	
\clearpage

\begin{figure}[ht]
\centering	
	\includegraphics[width=0.65\linewidth]{question2/luma_subsampled}
	\caption{Luma subsampled image}
\end{figure}


\appendix
\newpage

\section{Matlab Code}

\subsection{WorkHorse Function (Calls other functions)}
\lstinputlisting[language=Matlab]{"matlabFiles/lab1.m"}

\subsection{Generates normally distributed data}
\lstinputlisting[language=Matlab]{"matlabFiles/gaussTransform.m"}

\subsection{MED Classifier}
\lstinputlisting[language=Matlab]{"matlabFiles/MED_Class.m"}

\subsection{GED Classifiers}
\subsubsection{GED 2 Class}
\lstinputlisting[language=Matlab]{"matlabFiles/GED_Class2.m"}
\subsubsection{GED 3 Class}
\lstinputlisting[language=Matlab]{"matlabFiles/GED_Class3.m"}





\subsection{Classification grid prep}
\lstinputlisting[language=Matlab]{"matlabFiles/gridPrep.m"}

\subsection{Error Analysis}
\lstinputlisting[language=Matlab]{"matlabFiles/errorAnal.m"}








% -------- Bibliography --------
%\addcontentsline{toc}{chapter}{\hspace{13pt} References}
\bibliography{refs}

\end{document}  