\section{Generating Clusters}
Clusters were generated by sampling from a Gaussian distribution, then transforming the samples of each cluster based on the cluster's mean and covariance matrix. The random samples of a standard two-variable Gaussian distribution were generated using MATLAB's $randn()$ function. The samples were then transformed as shown by the example in MATLAB's documentation for $randn()$. The example shows to transform a given vector, x, by the equation
\begin{eqnarray}
\label{eqn:chol_transform}
\underbar {y} = chol(\Sigma) \underbar {x} + \underbar {$\mu$}
\end{eqnarray}

where $\underbar{x}$ is a vector in the original space, $\underbar{y}$ is the transformed vector in the new space, and $chol(\Sigma)$ returns the Cholesky factorization of $\Sigma$.

Unit standard deviation contours were also created for each cluster. To find these contours, a vector of points that represented the unit standard deviation contour of a standard two-variable Gaussian distribution was transformed using the previously described transformation method for each cluster.

This method was confirmed to be a viable method for transforming the data by using a second transformation to derive the unit standard deviation contours. The orthonormal covariance transformation shows that vectors of a Gaussian distribution with covarance $\Sigma$ can be transformed to have a covariance of I. This transformation is given by the equation
\begin{eqnarray}
\label{eqn:ortho_transform}
\underbar{x} = {\Lambda}^{-1/2} \Phi^T \underbar{y}
\end{eqnarray}

where columns of $\Phi$ are the eigenvectors of $\Sigma$ and the diagonal elements of $\Lambda$ are the eigenvectors of $\Sigma$. Taking the inverse of this equation and adding the specified mean results in the second transformation for deriving unit standard deviation contours:
\begin{eqnarray}
\label{eqn:ortho_inv_transform}
\underbar{y} = {(\Phi^{T})}^{-1} \Lambda^{-1/2}\underbar{x}+\underbar{$\mu$}
\end{eqnarray}

Both transformations produced the same unit standard deviation contours.
 
 \subsection{Classes A and B}
 The two classes A and B were characterized by:

\begin{eqnarray}
{\mu}_{A}=\left[ \begin{smallmatrix} 5&10 \end{smallmatrix}\right]^{T} \; & {\Sigma}_{A}=\left[ \begin{smallmatrix} 8&0 \\ 0&4 \end{smallmatrix}\right]^{T} \nonumber\\
{\mu}_{B}=\left[ \begin{smallmatrix} 10&15 \end{smallmatrix}\right]^{T} \; & {\Sigma}_{B}=\left[ \begin{smallmatrix} 8&0 \\ 0&4 \end{smallmatrix}\right]^{T} \nonumber
\end{eqnarray}

\begin{figure}[ht]
\centering
	\includegraphics[width=0.9\linewidth]{fig1a-AB_cluster}
	\label{fig:clustersDataAB}
	\caption{Clusters A and B with unit standard deviation ellipses}
\end{figure}

 \subsection{Classes C, D, E}
 The three classes C, D and E were characterized by:
 
 \begin{eqnarray}
{\mu}_{C}=\left[ \begin{smallmatrix} 5&10 \end{smallmatrix}\right]^{T} \; & {\Sigma}_{C}=\left[ \begin{smallmatrix} 8&4 \\ 4&40 \end{smallmatrix}\right]^{T} \nonumber\\
{\mu}_{D}=\left[ \begin{smallmatrix} 15&10 \end{smallmatrix}\right]^{T} \; & {\Sigma}_{D}=\left[ \begin{smallmatrix} 8&0 \\ 0&8 \end{smallmatrix}\right]^{T} \nonumber\\
{\mu}_{E}=\left[ \begin{smallmatrix} 10&5 \end{smallmatrix}\right]^{T} \; & {\Sigma}_{D}=\left[ \begin{smallmatrix} 10&-5 \\ -5&20 \end{smallmatrix}\right]^{T} \nonumber
\end{eqnarray}
 
 
\begin{figure}[ht]
\centering
	\includegraphics[width=0.9\linewidth]{fig1b-CDE_cluster}
	\label{fig:clustersDataCDE}
	\caption{Clusters C, D, E unit standard deviation ellipses}
\end{figure}
 
\clearpage
