\section{Error Analysis}
\subsection{Experimental error rate}
To find the experimental error rate $P(\epsilon)$ for Case 1, a discretized form of the following equation was used.
\begin{eqnarray}
\label{eqn:2-class_cont_PofE}
P(\epsilon)=\int_{R_A}P(\underbar {x}|B)P(B)d\underbar {x} +
            \int_{R_B}P(\underbar {x}|A)P(A)d\underbar {x}
\end{eqnarray}

The discretized form becomes
\begin{eqnarray}
\label{eqn:2-class_disc_PofE}
P(\epsilon)\approx\sum_{i,j\ in\ R_A}P(\underbar {x}|B)_{i,j}P(B)wh +
                  \sum_{i.j\ in\ R_B}P(\underbar {x}|A)_{i,j}P(A)wh
\end{eqnarray}

where $w$ and $l$ are the width and height of the discretized points. The MAP classification boundary that were found previously were used for determining $R_A$ and $R_B$. Although the equation describes summing values over an infinite number of points, a finite set of points had to be selected to feasibly perform the calculation. This introduces some negligible error. The same number of points used to calculate the classification data was used to calculate $P(\epsilon)$. The result was $P(\epsilon)\approx0.06281$.

To find $P(\epsilon)$ for Case 2, the same method was used as in Case 1 except that the equation for $P(\epsilon)$ had to be modified for three classes.
\begin{eqnarray}
\label{eqn:3-class_cont_PofE}
P(\epsilon)& = \int_{R_C}[P(\underbar{x}|D)P(D)+P(\underbar{x}|E)P(E)]d\underbar{x} \\
           & + \int_{R_D}[P(\underbar{x}|C)P(C)+P(\underbar{x}|E)P(E)]d\underbar{x} \nonumber\\
           & + \int_{R_E}[P(\underbar{x}|C)P(C)+P(\underbar{x}|D)P(D)]d\underbar{x}\nonumber
\end{eqnarray}

The discretized form of this equation becomes
\begin{eqnarray}
\label{eqn:3-class_disc_PofE}
P(\epsilon)&\approx \sum_{i,j\ in\ R_C}[P(\underbar{x}|D)_{i,j}P(D)+P(\underbar{x}|E)_{i,j}P(E)]wh \\
           & +      \sum_{i,j\ in\ R_D}[P(\underbar{x}|C)_{i,j}P(C)+P(\underbar{x}|E)_{i,j}P(E)]wh \nonumber\\
           & +      \sum_{i,j\ in\ R_E}[P(\underbar{x}|C)_{i,j}P(C)+P(\underbar{x}|D)_{i,j}P(D)]wh \nonumber
\end{eqnarray}

The result for Case 2 was found to be $P(\epsilon)\approx0.1686$.

Case 1 was found to have a much lower experimental error rate than Case 2. This was expected because Case 2 had one more class than Case 1 and because the unit standard deviation contours of the clusters of Case 2 were much closer together than the contours of the clusers of Case 1.
 
\subsection{Confusion matrices for GED, MED and MAP}

A confusion matrix shows the performance of a classifier on a set of data. It compares the true values of the data against the predicted classifications by the classifier, and attempts to quantify the types of errors being made.

For the GED, MED and MAP classifiers, the same data used to generate the classifier was used to test the classifier.

\subsubsection{GED}
\begin{figure}[!ht]
\begin{minipage}[b]{0.5\linewidth}
\centering
	\begin{tabular}{ccc|c|c}
	 & &\multicolumn{2}{c}{Predicted} &\\
	  & & \bf{A} &  \bf{B} & total \\
	 \cline{3-5}
	 \multirow{2}{*}{\begin{sideways}Actual\end{sideways}} & \bf{A'}& 186 & 14 & 200 \\
	 \cline{3-5}
	 & \bf{B'}& 10 & 190 & 200 \\
	  \cline{3-5}
	 &total&196&204&\\
	\end{tabular}
\end{minipage}
\hspace{0.5cm}
\begin{minipage}[b]{0.5\linewidth}
	\begin{tabular}{r|c}
	\hline
	Correct& 376\\
	Total& 400\\
	\hline
	\% Correct& 94\%\\
	\hline
	\end{tabular}
\end{minipage}
\vspace{1mm}
\caption{GED performance on A and B}
\end{figure}

\begin{figure}[!ht]
\begin{minipage}[b]{0.5\linewidth}
\centering
	\begin{tabular}{ccc|c|c|c}
	 & &\multicolumn{3}{c}{Predicted} &\\
	  & & \bf{C} &  \bf{D} & \bf{E} & total \\
	 \cline{3-6}
	 \multirow{3}{*}{\begin{sideways}Actual\end{sideways}} & \bf{C'}& 91 & 0 & 9 & 100\\
	 \cline{3-6}
	 & \bf{D'}& 6 & 164 & 30 & 200\\
	  \cline{3-6}
	 & \bf{E'}& 29 & 11 & 110 &  150\\
	  \cline{3-6}
	 &total&126&175&149\\
	\end{tabular}
\end{minipage}
\hspace{0.5cm}
\begin{minipage}[b]{0.5\linewidth}
	\begin{tabular}{r|c}
	\hline
	Correct& 365\\
	Total& 450\\
	\hline
	\% Correct& 81.1\%\\
	\hline
	\end{tabular}
\end{minipage}
\vspace{1mm}
\caption{GED performance on C, D and E}
\end{figure}

\clearpage

\subsubsection{MED}
\begin{figure}[!ht]
\begin{minipage}[b]{0.5\linewidth}
\centering
	\begin{tabular}{ccc|c|c}
	 & &\multicolumn{2}{c}{Predicted} &\\
	  & & \bf{A} &  \bf{B} & total \\
	 \cline{3-5}
	 \multirow{2}{*}{\begin{sideways}Actual\end{sideways}} & \bf{A'}& 184 & 16 & 200 \\
	 \cline{3-5}
	 & \bf{B'}& 10 & 190 & 200 \\
	  \cline{3-5}
	 &total&196&204&\\
	\end{tabular}
\end{minipage}
\hspace{0.5cm}
\begin{minipage}[b]{0.5\linewidth}
	\begin{tabular}{r|c}
	\hline
	Correct& 374\\
	Total& 400\\
	\hline
	\% Correct& 93.5\%\\
	\hline
	\end{tabular}
\end{minipage}
\vspace{1mm}
\caption{MED performance on A and B}
\end{figure}


\begin{figure}[!ht]
\begin{minipage}[b]{0.5\linewidth}
\centering
	\begin{tabular}{ccc|c|c|c}
	 & &\multicolumn{3}{c}{Predicted} &\\
	  & & \bf{C} &  \bf{D} & \bf{E} & total \\
	 \cline{3-6}
	 \multirow{3}{*}{\begin{sideways}Actual\end{sideways}} & \bf{C'}& 70 & 1 & 29 & 100\\
	 \cline{3-6}
	 & \bf{D'}& 9 & 167 & 24 & 200\\
	  \cline{3-6}
	 & \bf{E'}& 27 & 13 & 110 &  150\\
	  \cline{3-6}
	 &total&106&181&163\\
	\end{tabular}
\end{minipage}
\hspace{0.5cm}
\begin{minipage}[b]{0.5\linewidth}
	\begin{tabular}{r|c}
	\hline
	Correct& 347\\
	Total& 450\\
	\hline
	\% Correct& 77.1\%\\
	\hline
	\end{tabular}
\end{minipage}
\vspace{1mm}
\caption{MED performance on C, D and E}
\end{figure}

\subsubsection{MAP}
\begin{figure}[!ht]
\begin{minipage}[b]{0.5\linewidth}
\centering
	\begin{tabular}{ccc|c|c}
	 & &\multicolumn{2}{c}{Predicted} &\\
	  & & \bf{A} &  \bf{B} & total \\
	 \cline{3-5}
	 \multirow{2}{*}{\begin{sideways}Actual\end{sideways}} & \bf{A'}& 186 & 14 & 200 \\
	 \cline{3-5}
	 & \bf{B'}& 10 & 190 & 200 \\
	  \cline{3-5}
	 &total&196&204&\\
	\end{tabular}
\end{minipage}
\hspace{0.5cm}
\begin{minipage}[b]{0.5\linewidth}
	\begin{tabular}{r|c}
	\hline
	Correct& 376\\
	Total& 400\\
	\hline
	\% Correct& 94\%\\
	\hline
	\end{tabular}
\end{minipage}
\vspace{1mm}
\caption{MAP performance on A and B}
\end{figure}


\begin{figure}[!ht]
\begin{minipage}[b]{0.5\linewidth}
\centering
	\begin{tabular}{ccc|c|c|c}
	 & &\multicolumn{3}{c}{Predicted} &\\
	  & & \bf{C} &  \bf{D} & \bf{E} & total \\
	 \cline{3-6}
	 \multirow{3}{*}{\begin{sideways}Actual\end{sideways}} & \bf{C'}& 79 & 0 & 21 & 100\\
	 \cline{3-6}
	 & \bf{D'}& 4 & 174 & 22 & 200\\
	  \cline{3-6}
	 & \bf{E'}& 23 & 20 & 107 &  150\\
	  \cline{3-6}
	 &total&106&194&150\\
	\end{tabular}
\end{minipage}
\hspace{0.5cm}
\begin{minipage}[b]{0.5\linewidth}
	\begin{tabular}{r|c}
	\hline
	Correct& 360\\
	Total& 450\\
	\hline
	\% Correct& 80\%\\
	\hline
	\end{tabular}
\end{minipage}
\vspace{1mm}
\caption{MAP performance on C, D and E}
\end{figure}

\clearpage

\subsection{NN and 5NN}

For the NN and 5NN classifiers, the classifiers were developed from a set of training data, and the classifier was tested with a new set of data, distributed with the same underlying statistics as the training data. This data set was of the same size as the training data set.

\subsubsection{NN}
\begin{figure}[!ht]
\begin{minipage}[b]{0.5\linewidth}
\centering
	\begin{tabular}{ccc|c|c}
	 & &\multicolumn{2}{c}{Predicted} &\\
	  & & \bf{A} &  \bf{B} & total \\
	 \cline{3-5}
	 \multirow{2}{*}{\begin{sideways}Actual\end{sideways}} & \bf{A'}& 83 & 17 & 100 \\
	 \cline{3-5}
	 & \bf{B'}& 21 & 179 & 200 \\
	  \cline{3-5}
	 &total&104&196&\\
	\end{tabular}
\end{minipage}
\hspace{0.5cm}
\begin{minipage}[b]{0.5\linewidth}
	\begin{tabular}{r|c}
	\hline
	Correct& 279\\
	Total& 400\\
	\hline
	\% Correct& 69.8\%\\
	\hline
	\end{tabular}
\end{minipage}
\vspace{1mm}
\caption{Nearest Neighbour performance on A and B}
\end{figure}


\begin{figure}[!ht]
\begin{minipage}[b]{0.5\linewidth}
\centering
	\begin{tabular}{ccc|c|c|c}
	 & &\multicolumn{3}{c}{Predicted} &\\
	  & & \bf{C} &  \bf{D} & \bf{E} & total \\
	 \cline{3-6}
	 \multirow{3}{*}{\begin{sideways}Actual\end{sideways}} & \bf{C'}& 63 & 2 & 35 & 100\\
	 \cline{3-6}
	 & \bf{D'}& 3 & 160 & 37 & 200\\
	  \cline{3-6}
	 & \bf{E'}& 20 & 28 & 102 &  150\\
	  \cline{3-6}
	 &total&86&190&141\\
	\end{tabular}
\end{minipage}
\hspace{0.5cm}
\begin{minipage}[b]{0.5\linewidth}
	\begin{tabular}{r|c}
	\hline
	Correct& 325\\
	Total& 450 \\
	\hline
	\% Correct& 72.2\%\\
	\hline
	\end{tabular}
\end{minipage}
\vspace{1mm}
\caption{Nearest Neighbour performance on C, D and E}
\end{figure}

\clearpage

\subsubsection{5NN}
\begin{figure}[!ht]
\begin{minipage}[b]{0.5\linewidth}
\centering
	\begin{tabular}{ccc|c|c}
	 & &\multicolumn{2}{c}{Predicted} &\\
	  & & \bf{A} &  \bf{B} & total \\
	 \cline{3-5}
	 \multirow{2}{*}{\begin{sideways}Actual\end{sideways}} & \bf{A'}& 94 & 6 & 100 \\
	 \cline{3-5}
	 & \bf{B'}& 34 & 166 & 200 \\
	  \cline{3-5}
	 &total&128&172\\
	\end{tabular}
\end{minipage}
\hspace{0.5cm}
\begin{minipage}[b]{0.5\linewidth}
	\begin{tabular}{r|c}
	\hline
	Correct& 260\\
	Total& 300\\
	\hline
	\% Correct& 86.7\%\\
	\hline
	\end{tabular}
\end{minipage}
\vspace{1mm}
\caption{5 Nearest Neighbour performance on A and B}
\end{figure}

	
\begin{figure}[!ht]
\begin{minipage}[b]{0.5\linewidth}
\centering
	\begin{tabular}{ccc|c|c|c}
	 & &\multicolumn{3}{c}{Predicted} &\\
	  & & \bf{C} &  \bf{D} & \bf{E} & total \\
	 \cline{3-6}
	 \multirow{3}{*}{\begin{sideways}Actual\end{sideways}} & \bf{C'}& 75 & 0 & 25 & 100\\
	 \cline{3-6}
	 & \bf{D'}& 1 & 181 & 18 & 200\\
	  \cline{3-6}
	 & \bf{E'}& 14 & 24 & 112 &  150\\
	  \cline{3-6}
	 &total&90&205&155\\
	\end{tabular}
\end{minipage}
\hspace{0.5cm}
\begin{minipage}[b]{0.5\linewidth}
	\begin{tabular}{r|c}
	\hline
	Correct& 368\\
	Total& 450\\
	\hline
	\% Correct& 81.8\%\\
	\hline
	\end{tabular}
\end{minipage}
\vspace{1mm}
\caption{5 Nearest Neighbour performance on C, D and E}
\end{figure}

The confusion matrices of Case 2 show that points that actually belong to cluster E are the least likely to be classified correctly. Similar to why the experimental error rate for Case 2 was lower than Case 1, the greater difficulty in correctly classifying points of cluster E is because of how close cluster E is to the other two clusters. Cluster E is partially between clusters C and D. The GED and MAP classifier boundaries show this close proximity very well.
 
\clearpage
